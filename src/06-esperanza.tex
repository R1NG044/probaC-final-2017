\section{Vectores Aleatorios}
\begin{enumerate}
	\setcounter{enumi}{42}
	\item
		Sea $Y = g(X)$.
		\begin{align*}
			E(Y)	& = \sum_{y} y\cdot p(Y=y)								\\
					& = \sum_{y} \sum_{\{x : g(x) = y\}} y\cdot p(X=x)		\\
					& = \sum_{y} \sum_{\{x : g(x) = y\}} g(x)\cdot p(X=x)	\\
					& = \sum_{x} g(x)\cdot p(X=x)
		\end{align*}
		
		Cuando reemplazo $y=g(x)$, es importante que en cada término es potencialmente un $x$ distinto,
		pero como cada $x$ es de la preimagen de $g$ en $y$, las expresiones son equivalentes.
		
	\item
		$$E(X) = \int_0^{+\infty} x\cdot f_X(x)$$
		Hago partes usando el reemplazo $du = f_X(x)dx$, $u = -(1 - F_X(x))$ (o sea, hago aparecer lo que quiero en la integral de partes).
		
		Quedan $v = x$, $dv = dx$.
		\begin{align*}
			E(X)	& = \int_0^{+\infty} x\cdot f_X(x)	\\
					& = -x(1-F_X(x))\Big|_0^\infty - \int_0^{+\infty} -(1 - F_X(x))dx	\\
					& = -x(1-F_X(x))\Big|_0^\infty + \int_0^{+\infty} 1 - F_X(x)dx
		\end{align*}
		
		Queda ver que $x(1-F_X(x))\big|_0^\infty = 0$. Para $x=0$ se ve reemplazando, falta ver qué pasa cuando $x\rightarrow +\infty$.
		Pero,
		$$0 \leq x(1-F_X(x)) = x\int_x^{+\infty}f_X(u)du \leq \int_x^{+\infty}u\cdot f_X(u)du$$
		
		Como la esperanza está acotada, la parte de la derecha tiende a $0$ cuando $x\rightarrow +\infty$.

\end{enumerate}
