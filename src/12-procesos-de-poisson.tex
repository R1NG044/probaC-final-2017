\section{Procesos de Poisson}
\begin{enumerate}
	\setcounter{enumi}{88}
	\item
		$Y_n \sim \Gamma(n, \lambda)$ por ser suma de $\varepsilon(\lambda)$.
		\begin{align*}
			N(s) \geq n	& \Longleftrightarrow \{\text{Ocurrieron n o más eventos hasta s}\}			\\
						& \Longleftrightarrow \{\text{El evento n-ésimo no ocurrió después que s}\}	\\
						& \Longleftrightarrow Y_n \leq s
		\end{align*}
		Por la dualidad Poisson-Gamma, $N(s)\sim P(\lambda s)$.
	
	\item
		Sean $\tilde\tau_1, \cdots, \tilde\tau_n, \cdots$ los tiempos entre eventos consecutivos a partir del tiempo $s$. Sea $\tilde Y_n = \sum_{i=1}^n \tilde\tau_i$.
		
		Dados $N(s) = n$ y $Y_n = u$ tal que $u<s$, tenemos que $\tilde\tau_1 = Y_{n+1}-s$, y que para $i>1$ se cumple $\tilde\tau_i = \tau{n+i}$.
		
		Averigüemos la distribución de $\tilde\tau_1$:
		\begin{align*}
			P(\tilde\tau_1 > t_1 | N(s)=n \land Y_n = u)	& = P(Y_{n+1} - s > t_1 | Y_{n+1} \geq s \land Y_n = u)						\\
															& = P(Y_n + \tau_{n+1} - s > t_1 | Y_{n+1} \geq s \land Y_n = u)			\\
															& = P(\tau_{n+1} > t_1 + (s - u) | \tau_{n+1} + Y_n \geq s \land Y_n = u)	\\
															& = P(\tau_{n+1} > t_1 + (s - u) | \tau_{n+1} \geq (s - u))					\\
															& = P(\tau_{n+1} > t_1) = e^{-\lambda t_1}
		\end{align*}
		Luego $\tilde\tau_1 \sim \varepsilon(\lambda)$.
		
		Para el resto,
		
		\begin{align*}
			P\left(\bigcap_{i=1}^j \tilde\tau_i > t_i | N(s)=n \land Y_n = u\right)	& = P\left(\tilde\tau_1 > t_1 \land \bigcap_{i=2}^j \tilde\tau_i > t_i | N(s)=n \land Y_n = u\right)	\\
																					& = P\left(\tilde\tau_1 > t_1 \land \bigcap_{i=2}^j \tau_{n+i} > t_i | N(s)=n \land Y_n = u\right)		\\
																					& = e^{-\lambda t_1} \cdot \prod_{i=2}^j e^{-\lambda t_i} = e^{-\lambda \sum_{i=1}^j t_i}
		\end{align*}
		 Hacienmdo esto para cada $j$ se obtiene que $\tilde\tau_i \sim \varepsilon(\lambda)$, por lo que ambos procesos tienen la misma distribución.
		 
		 Son independientes porque el cálculo de $\tilde\tau_i$ depende sólo de $Y_{N(s)}$, y vimos que estas son independientes.
	\item
		Sea $N(t) \sim P(\lambda t)$. Sea $T_i$ el tiempo hasta la i-ésima llegada.
		
		\begin{align*}
			F_{T_i}(t_i)	& = P(T_i < t_i) = \sum_{j=i}^{+\infty} \frac{e^{-\lambda t_i}(\lambda t_i)^j}{j!}	\\
			f_{T_i}(t_i)	& = \sum_{j=i}^{+\infty}\frac{-\lambda e^{-\lambda t_i}(\lambda t_i)^j}{j!} + \sum_{j=i}^{+\infty}\frac{e^{-\lambda t_i}j(\lambda t_i)^{j-1}\lambda}{j!}	\\
							& = \sum_{j=i}^{+\infty}\frac{e^{-\lambda t_i}t_i^{j-1}\lambda^j}{(j-1)!} - \sum_{j=i}^{+\infty}\frac{e^{-\lambda t_i}t_i^j \lambda^{j+1}}{j!}				\\
							& = \sum_{j=i}^{+\infty}\frac{e^{-\lambda t_i}t_i^{j-1}\lambda^j}{\Gamma(j)} - \sum_{j=i+1}^{+\infty}\frac{e^{-\lambda t_i}t_i^{j-1}\lambda^j}{\Gamma(j)}	\\
							& = \frac{e^{-\lambda t_i}t_i^{i-1}\lambda^i}{\Gamma(i)}	\\
		\end{align*}
		Entonces $T_i \sim \Gamma(i, \lambda)$.
	\item
		$N(t) \sim P(3t) \Rightarrow N(3) \sim P(9)$.
		\begin{align*}
			P(N(3) \geq 3)	& = 1 - \sum_{i=0}^2 \frac{e^{-9} \cdot 9^i}{i!}	\\
							& = 1 - e^{-9}\left(1 + 9 + \frac{81}{2}\right)	\\
							& = 1 - e^{-9}\cdot\frac{101}{2} = 0.994
		\end{align*}
	\item
		\begin{enumerate}
			\item Propiedades de un proceso de Poisson:
				\begin{enumerate}
					\item $N(t) \sim P(\lambda t)$, $\forall t>0$. 
					\item Incrementos estacionarios: $N_{t+s} - N_s$ tiene $=$ distribución que $N_t$ (La cantidad de llegadas sólo depende del largo del intervalo).
					\item Incrementos independientes: Si dos intervalos $[a,b)$ y $[c,d)$ son disjuntos, $N_{b}-N{a}$ y $N_d - N_c$ son VA independientes.
				\end{enumerate}
			\item
				\begin{align*}
					P(AB)	& = P(A)P(B)					\\
							& = P(N_3 = 3)P(N_4 - N_3 = 4)	\\
							& = P(N_3 = 3)P(N_1 = 4)		\\
							& = \frac{e^{-3\lambda}\cdot(3\lambda)^3}{3!}\cdot \frac{e^{-\lambda}\cdot(\lambda)^4}{4!}
				\end{align*}
		\end{enumerate}
\end{enumerate}
