\section{Probabilidad. Definición y enunciados}
\begin{enumerate}
	\item Axiomas de probabilidad: Dados un experimento, y $S$ un espacio muestral, se asigna a cada evento $A$ una probabilidad $P(A)$ tal que:
	\begin{itemize}
		\item $P(A) \in [0,1]$
		\item $P(S) = 1$
		\item Si $A_1, A_2, \cdots$ disjuntos, $P(\cup_{i=0}^{\infty} A_i) = \sum_{i=0}^{\infty}P(A_i)$
	\end{itemize}
	\begin{enumerate}
		\item
			$$A \cup A^c = S$$ y aplicando probabilidad a ambos lados,
			$$P(A \cup A^c) = P(S)$$ 
			$$P(A) + P(A^c) = 1$$ pues $A$ y $A^c$ son disjuntos, y $P(S)=1$. 
			$$P(A) = 1 - P(A^c)$$ 
		\item 
			$$(A\setminus B) \cup B = A$$ porque $B\subset A$. Aplicando probabilidad:
			$$P((A\setminus B) \cup B) = P(A)$$
			$$P(A\setminus B) + P(B) = P(A)$$ pues $B$ y $A\setminus B$ son disjuntos.
			$$P(A\setminus B)= P(A) - P(B)$$
	\end{enumerate}
	\item 
	\item 
	\item 
	\item 
\end{enumerate}
