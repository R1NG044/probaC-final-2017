\section{Vectores Aleatorios}
\begin{enumerate}
	\setcounter{enumi}{72}
	\item
		Sea $\mathbb{I}_{[\epsilon, +\infty)}(X)$ la indicadora de $X\geq \epsilon$.
		$$\mathbb{I}_{[\epsilon, +\infty)}(X)\cdot \epsilon \leq X$$
		Porque si $X<\epsilon$ el lado izquierdo da $0$, y si no es obvio que $X\geq \epsilon$.
		
		Si tomamos esperanza a ambos lados, siendo que la esperanza de la indicadora es $P(X \geq \epsilon)$,
		$$P(X\geq\epsilon) \cdot \epsilon \leq E(X) \Rightarrow P(X\geq\epsilon) \leq \frac{E(X)}{\epsilon}$$
		
		Si tomamos la VA $(X - E(X))^2$ (sabemos que es $\geq 0$), y reescribimos la constante como $\epsilon^2$,
		\begin{align*}
			P((X - E(X))^2 \geq\epsilon^2)	& \leq \frac{E((X - \mu)^2)}{\epsilon^2}	\\
			P(|X - E(X)| \geq\epsilon)		& \leq \frac{V(X)}{\epsilon^2}
		\end{align*}
		
		Que es la fórmula de Chebychev.
		
	\item
		
\end{enumerate}
