\section{Vectores Aleatorios}
\begin{enumerate}
	\setcounter{enumi}{72}
	\item
		Sea $\mathbb{I}_{[\epsilon, +\infty)}(X)$ la indicadora de $X\geq \epsilon$.
		$$\mathbb{I}_{[\epsilon, +\infty)}(X)\cdot \epsilon \leq X$$
		Porque si $X<\epsilon$ el lado izquierdo da $0$, y si no es obvio que $X\geq \epsilon$.
		
		Si tomamos esperanza a ambos lados, siendo que la esperanza de la indicadora es $P(X \geq \epsilon)$,
		$$P(X\geq\epsilon) \cdot \epsilon \leq E(X) \Rightarrow P(X\geq\epsilon) \leq \frac{E(X)}{\epsilon}$$
		
		Si tomamos la VA $(X - E(X))^2$ (sabemos que es $\geq 0$), y reescribimos la constante como $\epsilon^2$,
		\begin{align*}
			P((X - E(X))^2 \geq\epsilon^2)	& \leq \frac{E((X - \mu)^2)}{\epsilon^2}	\\
			P(|X - E(X)| \geq\epsilon)		& \leq \frac{V(X)}{\epsilon^2}
		\end{align*}
		
		Que es la fórmula de Chebychev.
		
	\item
		La LGN débil dice que dada una sucesión de VA independientes $X_1, X_2, \cdots, X_n$
		con esperanza $\mu$ y varianza $\sigma^2 < \infty$, entonces
		$\overline X_n \xrightarrow{p} \mu$
		
		\textbf{Demostración:}
		
		Convergencia en probabilidad es que para cualquier $\varepsilon > 0$ se cumple
		$$\lim_{n\rightarrow\infty}P(|\overline X_n - \mu| > \varepsilon) = 0$$
		Primero, por linealidad de la esperanza, $E(\overline X_n) = \mu$.
		
		Entonces para probar lo de arriba, por Chebychev, basta ver que para $\varepsilon$ fijo,
		$$\lim_{n\rightarrow\infty}\frac{V(\overline X_n)}{\varepsilon^2} = 0$$
		Calculamos $V(\overline X_n)$:
		$$V(\overline X_n) = \frac{1}{n^2} \sum_{i=1}^n V(X_i) = \frac{\sum_{i=1}^n \sigma^2}{n^2} = \frac{\sigma^2}{n}$$
		Que efectivamente tiende a cero cuando $n\rightarrow\infty$.
	\item
		Primero verifico que $E(X_n) = \lambda$ y que $V(X_n) = \frac{\lambda}{n}$, y después aplicar Chebychev.
		Otra opción es decir que $X_n$ es la suma de $n$ VA independientes $P(\lambda)$, y usar LGN débil.
	\item
		Idem ejercicio anterior
	\item
		\begin{enumerate}
			\item
				$$X_n \sim U[0,1]$$
				$$Y_n = \text{mín}X_i$$
				$$U_n = n\cdot Y_n$$
				Quiero ver que $U_n \xrightarrow{d}W$ tal que $W\sim E(1)$.
				\begin{align*}
					1 - F_{U_n}(a)	& = P(U_n > a) = P\left(Y_n > \frac{a}{n}\right)	\\
									& = \prod_{i=1}^n P\left(X_n > \frac{a}{n}\right) = \prod_{i=1}^n \left(1 - \frac{a}{n}\right)	\\
									& = \left(1 - \frac{a}{n}\right)^n
				\end{align*}
				Entonces $$F_{U_n}(a) = 1 - \left(1 - \frac{a}{n}\right)^n$$
				\begin{align*}
					\lim_{n\rightarrow\infty}F_{U_n}(a)	& = \lim_{n\rightarrow\infty}1 - \left(1 - \frac{a}{n}\right)^n	\\
														& = \lim_{n\rightarrow\infty}1 - \left(1 - \frac{a}{n}\right)^{\left(-\frac{n}{a}\right)\left(-\frac{a}{n}\right)n}	\\
														& = \lim_{n\rightarrow\infty}1 - \left(1 - \frac{a}{n}\right)^{\left(-\frac{n}{a}\right)(-a)}	\\
														& = 1-e^{-a}
				\end{align*}
				Que es la acumulada de una exponencial con $\lambda=1$.
			\item
				$$X_n \sim U[0,1]$$
				$$Z_n = \text{máx}X_i$$
				$$V_n = n\cdot (1-Z_n)$$
				Quiero ver que $V_n \xrightarrow{d}W$ tal que $W\sim E(1)$.
				\begin{align*}
					1 - F_{V_n}(a)	& = P(V_n > a) = P\left(1-Z_n > \frac{a}{n}\right) = P\left(Z_n < 1 - \frac{a}{n}\right)			\\
									& = \prod_{i=1}^n P\left(X_i < 1 - \frac{a}{n}\right) =  \prod_{i=1}^n \left(1-\frac{a}{n}\right)	\\
									& = \left(1-\frac{a}{n}\right)^n
				\end{align*}
				El resto de la prueba es igual al caso anterior.
		\end{enumerate}
\end{enumerate}
