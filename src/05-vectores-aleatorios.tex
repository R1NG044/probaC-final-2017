\section{Vectores Aleatorios}
\begin{enumerate}
	\setcounter{enumi}{29}
	\item
		PREGUNTAR
	\item
		Sean $Z_1$ y $Z_2$ constantes de normalización de $f_{X_1}$ y $f_{X_2}$ respectivamente. $Z = Z_1\cdot Z_2$.
		$$P(X_1 = x_1 \land X_2 = x_2) = \frac{1}{Z}a^{x_1+x_2} = \frac{1}{Z_1}a^{x_1}\cdot \frac{1}{Z_2}a^{x_2}$$
		
		Por otro lado,
		\begin{align*}
			P(X_1 = x_1) & = \sum_{x_2 = 1}^{\infty} \frac{1}{Z_1}a^{x_1}\cdot \frac{1}{Z_2}a^{x_2} \\
			             & = \frac{1}{Z_1}a^{x_1} \sum_{x_2 = 1}^{\infty} \frac{1}{Z_2}a^{x_2} \\
			             & = \frac{1}{Z_1}a^{x_1} \cdot 1 \\
		\end{align*}
		(vale 1 porque es la sumatoria sobre toda la densidad de $X_2$).
		
		Análogamente, $P(X_2 = x_2) = \frac{1}{Z_2}a^{x_2}$. Como el producto de las densidades da la densidad conjunta, son independientes.
		
	\item
		\begin{align*}
			P(X_1 \geq x_1 \land X_2 \geq x_2)	& = a^{x_1 + x_2}			\\
			P(X_1 \geq x_1) = P(X_1 \geq x_1 \land X_2 \geq 0) = a^{x_1}
		\end{align*}
		
		Análogamente $P(X_2 \geq x_2) = a^{x_2}$.
		
	\item \hfill 
		\begin{itemize}				
			\item[($1\Rightarrow 2$)]
				Es fácil porque se define $g(x) = f_X(x)$ y $h(y) = f_Y(y)$.
			
			\item[($1\Rightarrow 3$)]
				Es lo mismo que el anterior, se define $G(x) = F_X(x)$ y $H(y) = F_Y(y)$.
				
			\item[($2\Rightarrow 1$)]
				\begin{align*}				
					f_{X,Y}(x,y)		& = g(x)h(y)			\\
					\int f_{X,Y}(x,y)dy	& = \int g(x)h(y) dy	\\
					f_X(x)				& = g(x) \int h(y) dy
				\end{align*}
				
				Es decir que $g(x) = C_x\cdot f_X(x)$ donde $C_x$ es una constante.
				
				Análogamente $h(y) = C_y\cdot f_Y(y)$ con $C_y$ constante.
				
				Entonces:
				$$f_{X,Y}(x,y) = C_xC_yf_X(x)f_Y(y)$$
				
				Y la constante vale 1 porque si integramos de los dos lados, $f_{X,Y}(x,y)$, $f_X(x)$ y $f_Y(y)$ son todas densidades e integran a 1.
				
				Finalmente $f_{X,Y}(x,y) = f_X(x)f_Y(y)$, con lo que son independientes.
				
			\item[($3\Rightarrow 2$)]
				\begin{align*}
					F_{X,Y}(x,y)									& = G(x)H(y)	\\
					\frac{\delta F_{X,Y}(x,y)}{\delta x}			& = G'(x)H(y)	\\
					\frac{\delta^2 F_{X,Y}(x,y)}{\delta x \delta y}	& = G'(x)H'(y)	\\
					f_{X,Y}(x,y)									& = G'(x)H'(y)
				\end{align*}
				
				Entonces basta definir $g(x) = G'(x)$ y $h(y) = H'(y)$ para demostrar (2).
		\end{itemize}
		
	\item
		$x_1 \cdots x_n \cdots \geq 0$ independientes con media $\mu$.
		
		Sea $N \in \{1\cdots k\}$ (supongo que equiprobable) independiente de todos los $x_i$.
		
		\begin{align*}
			E(\prod_{i=1}^N x_i)	& = \sum_{n=1}^k \frac{1}{k}E(\prod_{i=1}^n x_i) \\
									& = \frac{\sum_{n=1}^k \prod_{i=1}^n E(x_i)}{k} \\
									& = \frac{\sum_{n=1}^k (\mu ^n)}{k} \\
									& = \frac{\mu^{k+1} - 1}{(\mu - 1)k} \\
		\end{align*}
		
		En el primer paso, hago suma sobre todos los casos para $N$, por la probabilidad de cada uno (que es $\frac{1}{k}$). Puedo sacar la productoria para afuera en el paso 2 por linealidad de la esperanza.
		
		El último paso es la resolución analítica de esa sumatoria.
		
	\item
		$$P(R \leq a) = P(X^2 + Y^2 \leq a^2)$$
		
		Sean $r, \theta$ tales que $X = r\cdot sin(\theta)$ e $Y = r \cdot cos(\theta)$.
		Tenemos que $r^2 = X^2 + Y^2$.
		
		$$P(R \leq a) = P(r^2 < a^2) = P(|r| < a)$$
		suponiendo que $a > 0$.
		
		Quiero hallar:
		$$\int\int_D \frac{1}{2\pi} e^{\frac{-(x^2 + y^2)}{2}} dy\text{ }dx$$
		
		para $D = \{(x, y) \in \mathbb{R}^2 : x^2 + y^2 \leq a^2\}$.		
		
		Esto es lo mismo que integrar sobre:
		\begin{align*}
			\int\int_D \frac{1}{2\pi} e^{\frac{-(x^2 + y^2)}{2}} dy\text{ }dx
				& = \int_0^a\int_0^{2\pi} \frac{1}{2\pi} e^{\frac{-r^2}{2}} \cdot r d\theta\text{ }dr\\
				& = \int_0^a \left(\frac{1}{2\pi} e^{\frac{-r^2}{2}} \cdot r\right) \cdot 2\pi\text{ }dr \\
				& = \int_0^a (e^{\frac{-r^2}{2}} \cdot r) dr \\
				& = -\int_0^a ((-r)\cdot e^{\frac{-r^2}{2}}) dr \\
				& = -(e^{\frac{-a^2}{2}} -e^{\frac{-0^2}{2}}) \\
				& = 1 - e^{\frac{-a^2}{2}}
		\end{align*}
	
	\item
		$$F_{X+Y}(z) = P(X+Y \leq z) = \int_{-\infty}^{+\infty} \int_{-\infty}^{z-x} f_{X,Y}(x, y)\text{ }dy\text{ }dx$$
		Sean $u=x$, $v=x+y$:
		\begin{align*}
			F_{X+Y}(z)	& = \int_{-\infty}^{+\infty} \int_{-\infty}^{z} f_{X,Y}(u, v-u)\text{ }dv\text{ }du \\
						& = \int_{-\infty}^{z} \left(\int_{-\infty}^{+\infty} f_{X,Y}(u, v-u)\text{ }du\right)\text{ }dv
		\end{align*}
		Luego, lo de adentro del paréntesis es la densidad. Entonces:
		
		$$f_{X+Y}(z) = \int_{-\infty}^{+\infty} f_{X,Y}(u, z-u)\text{ }du$$
		
	\item
		Si reemplazamos la densidad por el producto de las densidades en la anterior:
		$$f_{X+Y}(z) = \int_{-\infty}^{+\infty} f_{X}(u)f_Y(z-u)\text{ }du$$
		
	\item
		Caso discreto:
		\begin{align*}
			P(g(X) = x' \land h(Y) = y')	& = \sum_{\{x:(g(x) = x')\}} \sum_{\{y:(h(y) = y')\}} P(X = x \land Y = y)	\\
											& = \sum_{\{x:(g(x) = x')\}} \sum_{\{y:(h(y) = y')\}} P(X = x) P(Y = y)		\\
											& = \left(\sum_{\{x:(g(x) = x')\}} P(X = x)\right) \left(\sum_{\{y:(h(y) = y')\}} P(Y = y)\right) \\
											& = P(g(X) = x') P(h(Y) = y')
		\end{align*}
		
		FALTA CASO CONTINUO PERO DEBE SER IGUAL, CON MENOR O IGUAL
		
	\item
		Sale haciendo el gráfico $(X,Y)$ y pintando el área que sirve, viendo que la proba de que no se crucen es como integrar un cuadrado de $\frac{5}{6}$ de lado.
		
	\item
		Sea $Y=aX+b$.
		Supongamos primero que $a>0$.
		
		$$F_{aX+b}(y) = P(aX+b \leq y) = P\left(X \leq \frac{y-b}{a}\right) = F_X\left(\frac{y-b}{a}\right)$$
		$$f_{aX+b}(y) = f_X\left(\frac{y-b}{a}\right)\frac{1}{a}$$
		
		Reemplazando en la densidad de la normal:
		\begin{align*}
			f_{aX+b}(y)	& = f_X\left(\frac{y-b}{a}\right)\frac{1}{a}																\\
						& = \frac{1}{\sqrt{2\pi \sigma^2}} e^{-\frac{\left(\frac{y-b}{a} - \mu\right)^2}{2\sigma^2}} \frac{1}{a}	\\
						& = \frac{1}{\sqrt{2\pi \sigma^2 a^2}} e^{-\frac{\left(\frac{y-b-a\mu}{a}\right)^2}{2\sigma^2}}				\\
						& = \frac{1}{\sqrt{2\pi (\sigma^2 a^2)}} e^{-\frac{y-(a\mu+b)}{2(\sigma^2a^2)}}
		\end{align*}
		
		Con lo que $Y \sim N(a\mu + b, a^2\sigma^2)$.
		
		Para el caso negativo, la ecuación con la $F$ se da vuelta, con lo que la $f$ queda multiplicada por $-1$,
		pero se cancela al meter el $\frac{1}{a}$ en la constante de la raíz porque $-\sqrt{a^2} = a$.
		
		Para ver la normalización, sea $Y = X - \mu$. $Y\sim N(0, \sigma^2)$. Luego sea $Z=\frac{Y}{\sigma^2}$. $Z\sim N(0,1)$.
	
	\item
		VER DESPUES
		
	\item
		VER DESPUES
\end{enumerate}
