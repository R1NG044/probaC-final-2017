\section{Cadenas de Markov}
\begin{enumerate}
	\setcounter{enumi}{93}
	\item
		Un proceso de Markov es un proceso estocástico (de tiempo discreto) tal que la probabilidad de $X_n$ sólo depende del valor de $X_{n-1}$ y de una matriz de transición $Q$.
		En esta matriz, $Q_{ij} = P(X_{n+1} = j | X_n = i)$.
		
		Para demostrar que $P(X_n = j | X_0 = i) = Q^n(i,j)$ procedemos por inducción.
		
		Para $n=1$ es cierto por la definición de $Q$.
		
		Supongamos que es cierto para todo número hasta $n$.
		
		\begin{align*}
			P(X_{n+1} = j | X_0 = i)	& = \sum_{k\in S} P(X_{n+1} = j | X_n = k \land X_0 = i) \cdot P(X_n = k | X_0 = i)	\\
										& = \sum_{k\in S} P(X_{n+1} = j | X_n = k) \cdot Q^n(i,k)							\\
										& = \sum_{k\in S} Q(k,j) \cdot Q^n(i,k)											\\
										& = Q^{n+1}(i,j)
		\end{align*}
		
	\item
		Demostrar que $Q^{n+m}(x,y) = \sum_{z\in S}Q^n(x,z) Q^m(z,y)$.
		
		\begin{align*}
			Q^{n+m}(x,y)	& = P(X_{n+m} = y | X_0=x)	\\
							& = \sum_{z\in S} P(X_{n+m} = y | X_n = z \land X_0=x) \cdot P(X_n = z | X_0=x)	\\
							& = \sum_{z\in S} P(X_{n+m} = y | X_n = z) \cdot Q^n(x,z)						\\
							& = \sum_{z\in S} Q^m(z,y) \cdot Q^n(x,z)
		\end{align*}
		
	\item
		La medida invariante es el autovector de autovalor 1 de $Q$.
		
		La matriz de transición es:
		$Q = \begin{bmatrix}
			0 & 1 & 0 & 0 & 0 \\
			1/4 & 0 & 3/4 & 0 & 0 \\
			0 & 1/2 & 0 & 1/2 & 0 \\
			0 & 0 & 3/4 & 0 & 1/4 \\
			0 & 0 & 0 & 1 & 0 \\
		\end{bmatrix}$
		
		Y planteando las ecuaciones derivadas de $\pi = \pi Q$ queda que $\pi = (1/16; 1/4; 3/8; 1/4; 1/16)$.
		
	\item
		La matriz de transición es:
		$$Q(x,y) = \frac{a(x,y)}{\text{gr}(x)}$$
		Donde $a(u,v)$ vale 1 si hay arista entre $u$ y $v$ y 0 si no, y donde $\text{gr}(x)$ es el grado de $x$.
		Como el grafo es no dirigido y conexo, el grado nunca es 0.
		
		Si a cada nodo le asigno probabilidad $\frac{\text{gr}(x)}{2A}$, siendo $A$ el número de aristas, ocurre que:
		$$\sum_{x}\frac{\text{gr}(x)}{2A} = \frac{1}{2A} \sum_{x}\text{gr}(x) = \frac{1}{2A} \cdot 2A = 1$$
		porque la suma de los grados recorre dos veces cada arista (una por cada uno de los vértices que conecta).
		
		Entonces es una probabilidad.
		Además,
		\begin{align*}
			P(X_{n+1} = y)	& = \sum_{\{x:(x,y) \in E\}}P(X_n = x) \cdot Q_{x,y}									\\
							& = \sum_{\{x:(x,y) \in E\}}\frac{\text{gr}(x)}{2A} \cdot \frac{a(x,y)}{\text{gr}(x)}	\\
							& = \frac{1}{2A}\sum_{\{x:(x,y) \in E\}} a(x,y)											\\
							& = \frac{\text{gr}(y)}{2A}
		\end{align*}
	
	\item
		Son dos urnas con $N$ bolitas en total, y en cada paso se elige entre las $N$ bolitas con probabilidad $\frac{1}{N}$ y se la cambia de urna.
		
		Ecuaciones de balance (para $0$, para $N$ y para $i/0<i<N$):
		$$\pi(0) = \frac{\pi(1)}{N}$$
		$$\pi(N) = \frac{\pi(N-1)}{N}$$
		$$\pi(i) = \frac{N-i+1}{N}\cdot \pi(i-1) + \frac{i+1}{N}\cdot \pi(i+1)$$
		
		Ahora verificamos que una $\text{Bi}(N, 1/2)$ cumple.
		
		$$\pi(0) = \binom{N}{0} \frac{1}{2^n} = \frac{1}{N}\cdot N \cdot \frac{1}{2^n} =
						\frac{1}{N}\cdot \binom{N}{1} \cdot \frac{1}{2^n} = \frac{\pi(1)}{N}$$
						
		$$\pi(N) = \binom{N}{N} \frac{1}{2^n} = \frac{1}{N}\cdot N \cdot \frac{1}{2^n} =
						\frac{1}{N}\cdot \binom{N}{N-1} \cdot \frac{1}{2^n} = \frac{\pi(N-1)}{N}$$
						
		\begin{align*}
			\pi(i)	& = \binom{N}{i} \frac{1}{2^N} = \left(\binom{N-1}{i} + \binom{N-1}{i-1}\right) \frac{1}{2^N}						\\
					& = \left(\frac{(N-1)!}{i!(N-i-1)!} + \frac{(N-1)!}{(i-1)!(N-i)!}\right) \frac{1}{2^N}								\\
					& = \left(\frac{i+1}{N}\cdot\frac{N!}{(i+1)!(N-i-1)!} + \frac{N-i+1}{N}\cdot\frac{N!}{(i-1)!(N-i+1)!}\right) \frac{1}{2^N}	\\
					& = \left(\frac{i+1}{N}\binom{N}{i+1} + \frac{N-i+1}{N}\binom{N}{i-1}\right) \frac{1}{2^N}							\\
					& = \frac{i+1}{N}\binom{N}{i+1}\frac{1}{2^N} + \frac{N-i+1}{N}\binom{N}{i-1}\frac{1}{2^N}							\\
					& = \frac{i+1}{N}\pi(i+1) + \frac{N-i+1}{N}\pi(i-1)
		\end{align*}
	
	\item
		
	\item
		Tenemos que $\pi = \pi Q$, y que $\pi(0) + \pi(1) = 1$:
		$$\pi(0) = 0.7\pi(0) + 0.4\pi(1) \Rightarrow \pi(0) = \frac{4}{3}\pi(1)$$
		Combinado con que suman 1, $$\pi = \left(\frac{4}{7}, \frac{3}{7}\right)$$
\end{enumerate}
