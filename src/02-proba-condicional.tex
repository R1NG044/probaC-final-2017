\section{Probabilidad condicional e independencia}
\begin{enumerate}
	\setcounter{enumi}{5}
	\item
		Sea E = la persona está enferma, y V = la persona está vacunada.
		La probabilidad pedida es $P(\overline V | E)$.
		$$P(\overline V | E) = \frac{P(\overline V \land E)}{P(E)} = \frac{0.13}{0.13+0.02} = \frac{13}{15}$$
	\item
		Asumimos que la probabilidad de que un hijo cualquiera sea varón es de $\frac{1}{2}$.
		Sea $V_1$ = el primer hijo es varón, y $V_2$ = el segundo hijo es varón.
		La probabilidad pedida es
		$$P(V_2|V_1) = \frac{P(V_1 \land V_2)}{P(V_1)} = \frac{1/4}{1/2} = \frac{1}{2}$$
	\item
		Dadas las mismas suposiciones, la probabilidad pedida es:
		$$P(V_1 \land V_2|V_1 \lor V_2) = \frac{P(V_1 \land V_2)}{P(V_1 \lor V_2)} = \frac{1/4}{1-1/4} = \frac{1}{3}$$
	\item
		Notamos cada caso como una tupla de dos letras (M o V), y ponemos un asterisco antes de la letra del hijo que abrió la puerta.
		
		Los casos posibles (que corresponden al evento en que abre la puerta un varón) son: $\{(V*V), (*VV), (*VM), (M*V)\}$.
		
		De estos, los favorables son los dos primeros (que corresponden al evento en que hay dos hijos varones). Por lo tanto la probabilidad es $\frac{2}{4} = \frac{1}{2}$.
	\item
		Sea $B_k = A_1A_2\cdots A_k$. Notar que $B_k = A_k B_{k-1}$ para $k>1$.
		
		Los voy a demostrar por inducción en $n$. La propiedad se puede enunciar como $$P(B_n) = P(A_1) P(A_2|B_1) \cdots P(A_n|B_{n-1})$$
		
		Caso base: $B_1 = A_1$. Esto ocurre por definición.
		
		Paso inductivo: vale la propiedad hasta $B_{n-1}$ y quiero demostrarla para $B_n$.
		
		$$P(A_n | B_{n-1}) = \frac{P(A_n B_{n-1})}{P(B_{n-1})} = \frac{P(B_n)}{P(B_{n-1})}$$
		$$P(B_n) = P(B_{n-1}) P(A_n | B_{n-1})$$
		
		Pero por HI, $$P(B_{n-1}) = P(A_1) P(A_2|B_1) \cdots P(A_{n-1}|B_{n-2})$$
		Luego tenemos
		$$P(B_n) = (P(A_1) P(A_2|B_1) \cdots P(A_{n-1}|B_{n-2})) P(A_n | B_{n-1})$$
		$$P(B_n) = P(A_1) P(A_2|B_1) \cdots P(A_n|B_{n-1})$$
		Q.E.D.
	\item
		$$P(B_1 = N \land B_3 = R) = \frac{4\cdot 3\cdot 5}{7\cdot 6\cdot 5} = \frac{2}{7}$$
	\item
		\begin{enumerate}
		\item Proba total:
			Sea $\{B_1, \cdots, B_n\}$ partición del espacio muestral $S$. Entonces $P(A) = \sum_{i=1}^{n} P(A|B_i)P(B_i)$.
			
			Demostración:
			
			$$P(A) = P(A \land S) = P\left(A \land \bigcup_{i=1}^{n}B_i\right) = P(\bigcup_{i=1}^{n}(A \land B_i)) = \sum_{i=1}^n P(A\land B_i)$$
			Porque los eventos $A\land B_i$ son disjuntos dos a dos.
			
			Finalmente, como $P(A|B) = P(A\land B)/P(B)$,
			$$P(A) = \sum_{i=1}^n P(A\land B_i) = \sum_{i=1}^n P(A|B_i)P(B_i)$$
		\item Bayes:
			Sea $\{B_1, \cdots, B_n\}$ partición del espacio muestral $S$. Entonces
				$$P(B_i|A) = \frac{P(A|B_i)P(B_i)}{P\left(\sum_{i=1}^{n} P(A|B_i)P(B_i)\right)}$$
				
			Demostración:
			
			El denominador, por Proba Total, es $P(A)$. Por otro lado, vale que:
			$$P(B_i|A)P(A) = P(A\land B) = P(A|B_i)P(B_i)$$
			Combinando los extremos y pasando $P(A)$ dividiendo se obtiene la demostración.
		\end{enumerate}
	\item
		Sin pérdida de generalidad el jugador elige la puerta 1. Sean $\{B_1, B_2, B_3\} = $ El premio está en la puerta $i$. Se cumple $P(B_i) = 1/3$.
		
		Si ocurre $B_1$, entonces conviene cambiar después de que el presentador abre alguna puerta. Si ocurre cualquiera de las otras dos no conviene.
		
		Por lo tanto la probabilidad de que convenga cambiar es $2/3$.
	\item
		Definición de independencia:
			
			Un conjunto de eventos $\{A_1, \cdots, A_n\}$ se dice independiente cuando, para cada $K$ subconjunto de $\{1, \cdots, n\}$, se cumple:
			$$P\left(\bigcap_{i\in K}{A_i}\right) = \prod_{i\in K}P(A_i)$$
		Ejemplo: Se tiran dos monedas y se definen las 3 variables aleatorias:
		\begin{itemize}
			\item $A = $ la primera moneda es cara
			\item $B = $ la segunda moneda es cara
			\item $C = $ las monedas son iguales
		\end{itemize}
		$A$ y $B$ son independientes porque son dos monedas diferentes. $A$ es independiente de $C$ porque $P(A|C) = P(A) = 1/2$. Sin embargo, cuando valen $A$ y $B$ a la vez, $P(C) = 1 \neq 1/2$, así que no son independientes.
	\item
		Sean $A$, $B$ eventos independientes, disitntos de $S$ y no vacíos. Entonces existen $a,b \in \{1, \cdots n-1\}$ tales que $P(A) = a/n$ y $P(B) = b/n$.
		
		Como son independientes, $$P(AB) = P(A)P(B) = \frac{ab}{n^2}$$
		Al mismo tiempo $AB$ es un evento de $S$, así que debe existir $c$ tal que $P(AB) = c/n$.
		
		Tenemos que $\frac{ab}{n^2} = \frac{c}{n}\Rightarrow ab=cn \Rightarrow n|ab$ (porque los 4 números son distintos de $0$).
		Como $n$ es primo, $n|a$ o $n|b$. Pero esto no puede ocurrir porque $0<a,b<n$. Entonces no pueden existir los eventos $A,B$. 
\end{enumerate}
