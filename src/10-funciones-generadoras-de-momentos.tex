\section{Funciones Generadoras de Momentos}
\begin{enumerate}
	\setcounter{enumi}{77}
	\item
		Quiero ver que $$M_{X_1+X_2+\cdots +X_n}(t) = \prod_{i=1}^n M_{X_i}(t)$$
		
		Como las $X_i$ son independientes, tenemos que
		$$M_{X_1+X_2+\cdots +X_n}(t) = E(e^{t\sum_{i=1}^n X_i}) = E(\prod_{i=1}^n e^{tX_i}) = \prod_{i=1}^n E(e^{tX_i}) = \prod_{i=1}^n M_{X_i}(t)$$
		
	\item
		$$M_{aX+b}(t) = E(e^{atX+bt}) = E(e^{atX})\cdot E(e^{bt}) = E(e^{atX})\cdot e^{bt} = M_{X}(at)\cdot e^{bt}$$
		$E(e^{bt}) = e^{bt}$  porque $e^{bt}$ es una constante.
		
	\item
		Quiero calcular $M_X(t) = E(e^{tX})$ para $X\sim N(0,1)$.
		
		$$E(e^{tX}) = \int_{-\infty}^{+\infty}e^{tX}\cdot f_X(x) dX = \int_{-\infty}^{+\infty}e^{tX}\cdot \frac{1}{\sqrt{2\pi}} e^{-\frac{X^2}{2}} dX$$
		$$ = \int_{-\infty}^{+\infty}\frac{1}{\sqrt{2\pi}} e^{tX-\frac{X^2}{2}} dX$$
		
		Ahora multiplico y divido por $e^{-\frac{t^2}{2}}$ para completar a una normal:
		
		$$ = e^{\frac{t^2}{2}} \cdot \int_{-\infty}^{+\infty}\frac{1}{\sqrt{2\pi}} e^{tX-\frac{X^2}{2}} \cdot e^{-\frac{t^2}{2}} dX$$
		$$ = e^{\frac{t^2}{2}} \cdot \int_{-\infty}^{+\infty}\frac{1}{\sqrt{2\pi}} e^{-\frac{1}{2}(X^2 - 2tX + t^2)} dX
		= e^{\frac{t^2}{2}} \cdot \int_{-\infty}^{+\infty}\frac{1}{\sqrt{2\pi}} e^{-\frac{(X-t)^2}{2}} dX$$
		
		Notamos que lo que está adentro de la integral es $f_X(x)$ si $X\sim N(t, 1)$. Por lo tanto integra a 1 :) y la FGM de la normal es:
		$$M_X(t) = e^{\frac{t^2}{2}}$$
		
	\item
		Sea $X \sim P(\lambda)$.
		
		$$M_X(t) = E(e^{tX}) = \sum_{k=0}^{\infty}e^{tk}\cdot\frac{e^{-\lambda}\cdot \lambda^k}{k!}
		= e^{-\lambda} \sum_{k=0}^{\infty}\frac{e^{tk}\lambda^k}{k!} = e^{-\lambda} \sum_{k=0}^{\infty}\frac{(e^t\lambda)^k}{k!}$$
		
		Ahora, sabemos que $\sum_{i=0}^n \frac{x^i}{i!} = e^x$ (se deriva de la f de la Poisson, de hecho).
		
		$$M_X(t) = e^{-\lambda} e^{e^t\lambda} = e^{e^t\lambda-\lambda} = e^{\lambda (e^t-1)}$$
		
		Ahora vamos a usar esto para calcular la FGM de una suma de Poissons.
		Si da la FGM de una Poisson, para cierto $\lambda$, demostramos que la suma es $P(\lambda)$ por unicidad de la FGM.
		
		Sean $X_i \sim P(\lambda_i)$ VA independientes.
		Como son independientes (por item 78):
		
		$$M_{\sum_{i=1}^n X_i}(t) = \prod_{i=1}^n M_{X_i}(t) = \prod_{i=1}^n e^{\lambda_i (e^t-1)} = e^{\sum_{i=1}^n \lambda_i (e^t-1)}
		= e^{(e^t-1) \sum_{i=1}^n \lambda_i}$$
		
		Por lo tanto, $\sum_{i=1}^n X_i \sim P(\sum_{i=1}^n \lambda_i)$.
\end{enumerate}
